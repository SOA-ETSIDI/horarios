%% https://github.com/tim6her/timetabel
\documentclass[10pt,a4paper, landscape, spanish]{article}
\usepackage[utf8]{inputenc} 
\usepackage[T1]{fontenc}
\usepackage[left=2cm,right=2cm,top=2cm,bottom=2cm]{geometry}
\usepackage[spanish]{babel}
\usepackage{eso-pic}
\newcommand\AtPageUpperRight[1]{\AtPageUpperLeft{%
   \makebox[\paperwidth][r]{#1}}}

\AddToShipoutPictureBG{%
  \AtPageUpperLeft{
    \raisebox{-1.1\height}{
      \includegraphics[width=0.12\textwidth]{LogoUPM.pdf}
    }}}

\AddToShipoutPicture{%
  \AtPageUpperRight{
    \raisebox{-1.1\height}{
      \includegraphics[width=0.12\textwidth]{LogoETSIDI.pdf}
    }}}

% Will change name of weekdays
\usepackage[spanish]{translator}

\pagestyle{empty}
\usepackage{tikz}
\usepackage{pgfcore}
\usepackage{pgfcalendar}
%% Disable hyphenation
\usepackage[none]{hyphenat}

\usepackage{mathpazo}
%% Colors as defined in ColorBrewer
\definecolor{Accent-5-1}{RGB}{127,201,127}
\definecolor{Accent-5-2}{RGB}{190,174,212}
\definecolor{Accent-5-3}{RGB}{253,192,134}
\definecolor{Accent-5-4}{RGB}{255,255,153}
\definecolor{Accent-5-5}{RGB}{56,108,176}

\definecolor{Set1-9-1}{RGB}{228,26,28}
\definecolor{Set1-9-2}{RGB}{55,126,184}
\definecolor{Set1-9-3}{RGB}{77,175,74}
\definecolor{Set1-9-4}{RGB}{152,78,163}
\definecolor{Set1-9-5}{RGB}{255,127,0}
\definecolor{Set1-9-6}{RGB}{255,255,51}
\definecolor{Set1-9-7}{RGB}{166,86,40}
\definecolor{Set1-9-8}{RGB}{247,129,191}
\definecolor{Set1-9-9}{RGB}{153,153,153}

% Setup of lengths
\newcommand{\widthofday}{4.9}
\newcommand{\lengthofhour}{1.1}
\newcommand{\entrytextwidth}{3.8}

% Set the first and last hour of the day (24h day format)
\newcommand{\firstH}{8}
\newcommand{\lastH}{21}


% Macros for drawing calendar entries, short version will only print the title
% of the entry 

% #1 Formato
% #2 Dia
% #3 HoraInicio
% #4 HoraFinal
% #5 Asignatura
% #6 Tipo de docencia
% #7 Aula (o grupo correspondiente para doble grado)
\newcommand{\calentry}[7][]{
	\filldraw[black!50, rounded corners, #1] 
		({day(#2,-1)},{time(#3) - 0.015}) rectangle ({day(#2,1)},{time(#4) + 0.015});%
	\node[anchor=north, text width = 1.1 * \entrytextwidth cm, #1]
        at ({day(#2,0)},{time(#3) - 0.016}) 
        {\footnotesize #3 -- #4 :: \emph{#6} \hfill \textbf{#7}\\ \small \textsc{#5}};%
}
\newcommand{\shortcalentry}[6][]{
	\filldraw[black!50, rounded corners, #1] 
		({day(#2,-1)},{time(#3) - 0.015}) rectangle ({day(#2,1)},{time(#4) + 0.015});%
	\node[font=\scriptsize, align = center, text width=\entrytextwidth cm, #1] 
		at ({day(#2,0)},{halftime(#3,#4)}) {#5};%
}


% Offset for the rectangle marking appointments
\pgfmathsetmacro{\offsetForAppointment}{+0.01pt}

% Converts the integer marking the day of the week to a coordinate.
% The second argument specifies wheter the left (-1), the base (0) or the right
% end of the column is returned.
\pgfmathdeclarefunction{day}{2}{%
	\pgfmathparse{-#2*\offsetForAppointment+#1-(1-#2)/2}%
}

% Converts a given time (hh.mm) to a length. \firstH is converted to 0.
\pgfmathdeclarefunction{time}{1}{%
	\pgfmathparse{\firstH-(floor(#1)+(#1-floor(#1))/0.6)}%
}

% Finds the mean of the duration of the appointment. Used in \shortcalentry.
\pgfmathdeclarefunction{halftime}{2}{%
	\pgfmathparse{(time(#1)+time(#2))/2}%
}

\usepackage{environ}
\NewEnviron{timetable}%
		[2]%
		{%
\renewcommand{\firstH}{#1}
\renewcommand{\lastH}{#2}
%
\begin{tikzpicture}[x=\widthofday cm, y=\lengthofhour cm]

% Draws the timtables grid
\foreach \hour in {\firstH,...,\lastH}
        \draw[line width = .5pt, gray] (0,\firstH-\hour) node[left=5] {\hour.00} -- (5,\firstH-\hour)
                 (0,\firstH+0.5-\hour) -- (5,\firstH+0.5-\hour);
\foreach \day in {0,...,5}
        \draw[line width=8pt,white] (\day,0.6) -- (\day,\firstH-\lastH -0.1);

% Prints days of the week, first day of the week (1) is Monday!
\foreach \day in {0,...,4}
        \node at (0.5+\day,1) {\pgfcalendarweekdayname{\day}};

{\BODY}

\end{tikzpicture}
		}%

% TikZ style to display different subcalendars
\tikzstyle{tut}=[fill=Accent-5-4!40!white, text=black]
\tikzstyle{free}=[fill=black!10, text=black]

% Formatos dependientes del tipo de docencia
\tikzstyle{TyP}=[fill=Accent-5-1!70!white, text=black]
\tikzstyle{Lab}=[fill=Accent-5-5!70!white, text=black]
\tikzstyle{AC}=[fill=Accent-5-3!70!white, text=black]
\tikzstyle{misc}=[fill=Accent-5-2!70!white, text=black]

% Formatos dependientes de la asignatura
\tikzstyle{A}=[fill=Set1-9-1!40!white, text=black]
\tikzstyle{B}=[fill=Set1-9-2!40!white, text=black]
\tikzstyle{C}=[fill=Set1-9-3!40!white, text=black]
\tikzstyle{D}=[fill=Set1-9-4!40!white, text=black]
\tikzstyle{F}=[fill=Set1-9-5!40!white, text=black]
\tikzstyle{E}=[fill=Set1-9-6!40!white, text=black]
\tikzstyle{G}=[fill=Set1-9-7!40!white, text=black]
\tikzstyle{H}=[fill=Set1-9-8!40!white, text=black]
\tikzstyle{I}=[fill=Set1-9-9!40!white, text=black]
\tikzstyle{J}=[fill=Set1-9-9!30!white, text=black]
\tikzstyle{K}=[fill=Set1-9-9!20!white, text=black]
\tikzstyle{L}=[fill=Set1-9-1!20!white, text=black]
\tikzstyle{M}=[fill=Set1-9-2!20!white, text=black]
\tikzstyle{N}=[fill=Set1-9-3!20!white, text=black]



